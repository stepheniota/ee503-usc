\documentclass{article}
\usepackage{geometry}

\def\nauthor{Stephen Iota}
\def\nclass{ee503 probability}
\def\npset{Problem Set 6}
\def\ndate{\today}
\def\nemail{iota@usc.edu}
\def\nthanks{SID: 6862013543}

%%%%%%%%%%%%%%%%%%%%%%%%%%%%%%%%%%%%%%%%%%%%%%%%%%%%%
%% LaTeX2e Template by Stephen Iota (iota@usc.edu) %%
%%%%%%%%%%%%%%%%%%%%%%%%%%%%%%%%%%%%%%%%%%%%%%%%%%%%%
\usepackage[utf8]{inputenc}
\usepackage[T1]{fontenc}
\usepackage{amsmath, amssymb, amsthm}
\usepackage{physics}
%\usepackage{algorithm}
%\usepackage[noend]{algorithmic}
\usepackage{mathtools}  % for boxed answers in align environments
%\usepackage{cancel}
\usepackage{graphicx}
\usepackage[shortlabels]{enumitem}  % change labels in enum/item envs, [noitem[list]sep]
%\usepackage[labelfont=bf,font=small]{caption}
\usepackage[dvipsnames]{xcolor}
%\usepackage[big]{titlesec}  % [small,medium,big]
%\usepackage{fancyhdr}
%\usepackage[noadjust]{cite}
%\usepackage{tikz}
\usepackage[colorlinks=true,
            citecolor=NavyBlue!90!black,
            linkcolor=green!50!black,
            urlcolor=green!50!black,
            hypertexnames=false]{hyperref}

%%%%%%%%%%%%%%%%%%%%%%%
%%%% MISC COMMANDS %%%%
%%%%%%%%%%%%%%%%%%%%%%%
\graphicspath{{./figures/}} % Setting the graphics path
\newcommand{\email}[1]{\texttt{\href{mailto:#1}{#1}}}
\newcommand{\hreftt}[2]{\texttt{\href{#1}{#2}}}
\newcommand{\pref}[1]{[\ref{#1}]}
\newcommand{\nextproblem}[0]{\bigbreak}


%%%%%%%%%%%%%%%%%%%%
%%% FRONT MATTER %%%
%%%%%%%%%%%%%%%%%%%%
\def\makemytitle{
	\begin{center}
        {\LARGE \textsc{\nclass}: \npset}%\textbf{\npset}}
    \end{center}
    \bigbreak
    \begin{center}
        \nauthor        \\
        \email{\nemail} \\
		\nthanks        \\
        \ndate
    \end{center}
}


%%%%%%%%%%%%%%%%%%
%% Environments %%
%%%%%%%%%%%%%%%%%%
%\numberwithin{equation}{section}
\theoremstyle{plain}
\newtheorem{problem}{Problem}
%\numberwithin{problem}{Problem}
\theoremstyle{definition}
%\swapnumbers % `2.1 Solution' instead of `Solution 2.1'
\newtheorem*{solution}{Solution}
%\numberwithin{solution}{solution}
\renewcommand\qedsymbol{$\blacksquare$}


%%%%%%%%%%%%%%%%%%%%%%%
%%%% MATH COMMANDS %%%%
%%%%%%%%%%%%%%%%%%%%%%%
\newcommand{\transpose}[1]{\ensuremath{#1^T}}
\newcommand{\colvec}[1]{\ensuremath{\begin{pmatrix} #1 \end{pmatrix}}}
\newcommand{\eye}[0]{\ensuremath{\mathbb{I}}}
\newcommand{\R}[0]{\ensuremath{\mathbb{R}}}
\newcommand{\union}[0]{\cup}
\newcommand{\intersect}[0]{\cap}
\newcommand{\set}[1]{\ensuremath{\{#1\}}}
\newcommand{\compl}[1]{\ensuremath{#1^{\complement}}}
\newcommand{\factorial}[1]{\ensuremath{#1!}}
\newcommand{\percentile}[0]{\ensuremath{\pi(r)}}
\newcommand{\ceil}[1]{\ensuremath{\lceil #1 \rceil}}
\newcommand{\floor}[1]{\ensuremath{\lfloor #1 \rfloor}}

\DeclareMathOperator{\expect}{E}
\DeclareMathOperator{\E}{E}
\let\P\relax
\DeclareMathOperator{\P}{P}



%%%%%%%%%%%%%
%%% Begin %%%
%%%%%%%%%%%%%
\begin{document}

    \makemytitle

    \nextproblem
    \nextproblem

    \begin{problem}
        An important manufacturing process produces cylindrical component parts for the automotive industry. It is important that the process produce parts having a mean diameter of $5$ mm. The engineer involved conjectures that the population mean is $5.0$ mm. An experiment is conducted in which $100$ parts produced by the process are selected randomly and the diameter measured on each. It is known that the population standard deviation $\sigma = 0.1$. The experiment indicates a sample average diameter $x = 5.027$ mm. Does this sample information appear to support or refute the engineer’s conjecture
    \end{problem}

    \begin{solution}
        For a random sample $x_1, x_2, \ldots, x_n$, the sample mean is given by $M_n = \frac{1}{n}\sum_i^n x_i$. Since the population variance is known, we can construct a confidence interval to see how well the sample mean represents the population mean, $m \in [M_n \mp \frac{\sigma y_{\alpha/2}}{\sqrt{n}}]$.

        Using a lookup table, it can be shown that
        \begin{equation}
            m = 5.027 \pm 0.02576 \text{ mm with } 99\% \text{ probability}.
        \end{equation}
        So our sample mean is not even in the $99\%$ confidence interval. Therefore our estimated population mean of $5.0$ mm is likely incorrect, and the true mean is likely much higher.
    \end{solution}

    \nextproblem

    \begin{problem}
        A random variable $X$ is believed to be uniformly distributed in the interval $[0,a]$, where $a$ is unknown. Derive the maximum likelihood estimate, $\hat{a}$, of $a$ from $n$ independent observations of $X$ labelled $X_1, \ldots, X_n$. Find the mean and variance of $\hat{a}$ and explain why it is a good estimate when n is large.
    \end{problem}

    \begin{solution}
    The pdf of the r.v. $X$ is given by
    \begin{equation*}
        f_{X|a}(x|a) = \begin{cases}
                        1/a & \text{for } x \in [0, a] \\
                        0 & \text{else}
                        \end{cases}
    \end{equation*}
    Therefore, the likelihood function $l$ is
    \begin{equation}
        l(\{X_i\}_n; a) = \begin{cases} 
        \frac{1}{a^n} & \text{ if } X_i \leq a \quad \forall X_i \in \{X_i\}_n \\
        0 & \text{ else}
        \end{cases}
    \end{equation}
    $l$ is an ever-decreasing function of $a$; its minimum w.r.t.~$a$ is in the limit towards infinity.
    Thus, it is clear that the value of $a$ which maximizes the likelihood function can by any $a$ such that
    \begin{equation}
        a \geq \max(X_i, \ldots, X_n)
    \end{equation}
    , otherwise the likelihood function would be zero.
    \end{solution}

    \nextproblem

    \begin{problem}
        A communication channel accepts as input either $000$ or $111$. The channel transmits each bit ($0$ or $1$) correctly with probability $1 - p$ and erroneously with probability $p$. Find the entropy of the input given that the output is $010$.
    \end{problem}

    \begin{solution}
        We are asked to find the entropy of the input $X$ given the output $Y = (010)$.
        \begin{align}
            H(X|Y=010) = -(p_X(000|010) \log p_X(000|010) + p_X(111|010) \log{p_X(111|010)})
        \end{align}
        where $p_X(000|010) = \frac{p(1-p)^2}{p(1-p)^2 + p^2(1-p)}$ and $p_X(111|010) = \frac{p^2(1-p)}{p(1-p)^2 +  p^2(1-p)}$.
    \end{solution}

    \nextproblem

    \begin{problem}
        A transmitter generates a set of messages $m \in \set{1, 2, 3, 4, 5, 6, 7, 8}$ with probabilities $\set{1/4, 1/4, 1/8, 1/8, 1/8, 1/16, 1/32, 1/32}$, respectively.
        \begin{enumerate}
            \item[(a)] What is the entropy of $m$?
            \item[(b)] For binary transmission, $n$ distinct messages can be represented using $\ceil{\log_2{n}}$ bits. E.g., the 8 messages in m can be represented uniquely using 3 bits with the following mapping $\set{1, 2, 3, 4, 5, 6, 7, 8} \rightarrow \set{001, 010, 011, 100, 101, 110, 111, 000}$. Such a mapping is called a \emph{code} and each 3-bit representation of a message is called a \emph{codeword}.  For the above code, the average length of a codeword is 3 bits since each codeword is 3 bits long. Can you design another code for m such that the average length of a codeword is equal to the entropy of m?
        \end{enumerate}
    \end{problem}

    \begin{solution}
        The entropy of some r.v.~$X$ is given by
        \begin{equation*}
            H(X) = -\sum_{x \in X}p_X(x) \log(p_X(x))
        \end{equation*}
    \end{solution}
    Considering the set of messages, the entropy of $m$ is equal to $\boxed{H(m) = 43/16 = 2 + 11/16}$.

    We can make a code whose average length is equal to $H(m)$ by the following. Let $n_i$ be the number of bits needed to represent message $m_i$.
    \begin{equation}
        n_i = \log{1/p_i}
    \end{equation}
    From here, it is easy to see the average number of bits needed is equal to $H(m)$.
    \begin{align*}
        E[n] &= E[\log{1/p_i}] \\
        &= \sum_i p_i \log{1/p_i} \\
        &= -\sum_i p_i \log{p_i} \\
        &= H(m)
    \end{align*}



\end{document}
utu