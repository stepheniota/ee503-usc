%%%%%%%%%%%%%%%%%%%%%%%%%%%%%%%%%%%%%%%%%%%%%%%%%%%%%
%% LaTeX2e Template by Stephen Iota (iota@usc.edu) %%
%%%%%%%%%%%%%%%%%%%%%%%%%%%%%%%%%%%%%%%%%%%%%%%%%%%%%
\usepackage[utf8]{inputenc}
\usepackage{amsmath, amssymb, amsthm}
\usepackage{physics}
%\usepackage{algorithm}
%\usepackage[noend]{algorithmic}
\usepackage{mathtools}  % for boxed answers in align environments
%\usepackage{cancel}
\usepackage{graphicx}
\usepackage[shortlabels]{enumitem}  % change labels in enum/item envs, [noitem[list]sep]
%\usepackage[labelfont=bf,font=small]{caption}
\usepackage[dvipsnames]{xcolor}
%\usepackage[big]{titlesec}  % [small,medium,big]
%\usepackage{fancyhdr}
%\usepackage[noadjust]{cite}
%\usepackage{tikz}
\usepackage[colorlinks=true,
            citecolor=NavyBlue!90!black,
            linkcolor=green!50!black,
            urlcolor=green!50!black,
            hypertexnames=false]{hyperref}

%%%%%%%%%%%%%%%%%%%%%%%
%%%% MISC COMMANDS %%%%
%%%%%%%%%%%%%%%%%%%%%%%
\graphicspath{{./figures/}} % Setting the graphics path
\newcommand{\email}[1]{\texttt{\href{mailto:#1}{#1}}}

\newcommand{\pref}[1]{[\ref{#1}]}

%%%%%%%%%%%%%%%%%%%%
%%% FRONT MATTER %%%
%%%%%%%%%%%%%%%%%%%%
\def\makemytitle{
	\begin{center}
        {\LARGE \textsc{\nclass}: \npset}%\textbf{\npset}}
    \end{center}
    \bigbreak
    \begin{center}
        \nauthor        \\
        \email{\nemail} \\
		\nthanks        \\
        \ndate
    \end{center}
}

%%%%%%%%%%%%%%%%%%%%%%%%%%%%%%%%
%% My commands & environments %%
%%%%%%%%%%%%%%%%%%%%%%%%%%%%%%%%
%\numberwithin{equation}{section}
\theoremstyle{plain}
\newtheorem{problem}{Problem}
%\numberwithin{problem}{Problem}
\theoremstyle{definition}
%\swapnumbers % `2.1 Solution' instead of `Solution 2.1'
\newtheorem*{solution}{Solution}
%\numberwithin{solution}{solution}
\renewcommand\qedsymbol{$\blacksquare$}

%%%%%%%%%%
%% MISC %%
%%%%%%%%%%
\newcommand{\nextproblem}[0]{\bigbreak}


%%%%%%%%%%%%%%%%%%%%%%%
%%%% MATH COMMANDS %%%%
%%%%%%%%%%%%%%%%%%%%%%%
\newcommand{\transpose}[1]{\ensuremath{#1^T}}
\newcommand{\colvec}[1]{\ensuremath{\begin{pmatrix} #1 \end{pmatrix}}}
\newcommand{\eye}[0]{\ensuremath{\mathbb{I}}}
\newcommand{\R}[0]{\ensuremath{\mathbb{R}}}
\newcommand{\union}[0]{\cup}
\newcommand{\intersect}[0]{\cap}
\newcommand{\set}[1]{\ensuremath{\{#1\}}}
\newcommand{\compl}[1]{\ensuremath{#1^{\complement}}}
\newcommand{\factorial}[1]{\ensuremath{#1!}}
\newcommand{\percentile}[0]{\ensuremath{\pi(r)}}